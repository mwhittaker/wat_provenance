\documentclass{mwhittaker}
\title{Wat-Provenance}
\author{~}
\date{~}

\usepackage{pervasives}
\usepackage{subcaption}

\begin{document}
\maketitle

\section{Relational Algebra Provenance}
\begin{figure}[h]
  \begin{subfigure}[b]{0.5\textwidth}
    \centering
    \[
      \begin{array}{rrl}
        Q & ::= & \emptyset \\
          &   | & \set{t} \\
          &   | & R \\
          &   | & \select_\phi(Q) \\
          &   | & \project_A(Q) \\
          &   | & Q_1 \join Q_2 \\
          &   | & Q_1 \cup Q_2 \\
          &   | & \rename_{A \mapsto B}(Q)
      \end{array}
    \]
    \caption{Relational algebra syntax}
    \label{fig:ra-syntax}
  \end{subfigure}
  \begin{subfigure}[b]{0.5\textwidth}
    \centering
    \[
      \begin{aligned}
        \denote{\emptyset}(I) &= \emptyset \\
        \denote{\set{t}}(I) &= \set{t} \\
        \denote{R}(I) &= I(R) \\
        \denote{\select_\phi(Q)}(I) &= \setst{t \in \denote{Q}(I)}{\phi(t)} \\
        \denote{\project_A(Q)}(I) &= \setst{t[A]}{t \in \denote{Q}(I)} \\
        \denote{Q_1 \join Q_2}(I) &= \setst{t}{t[A_1] \in \denote{Q_1}(I) \land t[A_2] \in \denote{Q_2}(I)} \\
        \denote{Q_1 \cup Q_2}(I) &= \denote{Q_1}(I) \cup \denote{Q_2}(I) \\
        \denote{\rename_{A \mapsto B}(Q)}(I) &= \setst{t[A \mapsto B]}{t \in \denote{Q}(I)}
      \end{aligned}
    \]
    \caption{Relational algebra semantics}
    \label{fig:ra-semantics}
  \end{subfigure}
\end{figure}

\subsection{Lineage}
\begin{align*}
  \Lineage(\emptyset, I, t) &= \bot \\
  \Lineage(\set{t}, I, u) &= \begin{cases}
    \emptyset & t = u \\
    \bot & t \neq u
  \end{cases} \\
  \Lineage(R, I, t) &= \begin{cases}
    \set{t} & t \in I(R) \\
    \bot & t \notin I(R)
  \end{cases} \\
  \Lineage(\select_\phi(Q), I, t) &= \begin{cases}
    \Lineage(Q, I, t) & \phi(t) \\
    \bot & \lnot \phi(t)
  \end{cases} \\
  \Lineage(\project_A(Q), I, t) &= \bigcup_L \setst{\Lineage(Q, I, u)}{u \in \denote{Q}(I), u[A] = t} \\
  \Lineage(Q_1 \join Q_2, I, t) &= \Lineage(Q_1, I, t[A_1]) \cup_S \Lineage(Q_2, I, t[A_2]) \\
  \Lineage(Q_1 \cup Q_2, I, t) &= \Lineage(Q_1, I, t) \cup_L \Lineage(Q_2, I, t) \\
  \Lineage(\rename_{A \mapsto B}(Q), I, t) &= \Lineage(Q, I, t[B \mapsto A])
\end{align*}

\subsection{Why-Provenance}
\begin{align*}
  \Why(\emptyset, I, t) &= \emptyset \\
  \Why(\set{t}, I, u) &= \begin{cases}
    \set{\emptyset} & t = u \\
    \emptyset & t \neq u
  \end{cases} \\
  \Why(R, I, t) &= \begin{cases}
    \set{\set{t}} & t \in I(R) \\
    \emptyset & t \notin I(R) \\
  \end{cases} \\
  \Why(\select_\phi(Q), I, t) &= \begin{cases}
    \Why(Q, I, t) & \phi() \\
    \emptyset & \lnot \phi(t)
  \end{cases} \\
  \Why(\project_A(Q), I, t) &= \bigcup \setst{\Why(Q, I, u)}{u \in \denote{Q}(I), u[A] = t} \\
  \Why(Q_1 \join Q_2, I, t) &= \Why(Q_1, I, t[A_1]) \Cup \Why(Q_2, t, t[A_2]) \\
  \Why(Q_1 \cup Q_2, I, t) &= \Why(Q_1, I, t) \cup \Why(Q_2, I, t) \\
  \Why(\rename_{A \mapsto B}(Q), I, t) &= \Why(Q, I, t[B \mapsto A])
\end{align*}

Some definitions:
\begin{align*}
  \Wit(Q, I, t)
    &\defeq \setst{J \subseteq I}{t \in \denote{Q}(J)} \\
  \MWit(Q, I, t)
    &\defeq \setst{J \in \Wit(Q, I, t)}{\text{$J$ is minimal in $\Wit(Q, I, t)$}} \\
  \MWhy(Q, I, t)
    &\defeq \setst{J \in \Why(Q, I, t)}{\text{$J$ is minimal in $\Why(Q, I, t)$}}
\end{align*}

Some theorems:
\begin{align*}
  \Wit(Q, I, t)
    &= \setst{J \subseteq I}{\exists J' \in \Wit(Q, I, t).\ J' \subseteq J} \\
  \MWhy(Q, I, t)
    &= \MWit(Q, I, t)
\end{align*}

\subsection{How-Provenance}
A \textbf{commutative semiring} is an algebraic structure $(K, +, \cdot, 0, 1)$
where
\begin{itemize}
  \item $(K, +, 0)$ is a commutative monoid,
  \item $(K, \cdot, 1)$ is a commutative monoid,
  \item multiplication distributes over addition, and
  \item 0 is an annihilator under multiplication.
\end{itemize}

Given a commutative semigroup $K$, a $K$-relation $R$ is a function from a
domain of tuples to $K$. The $K$-relational algebra has the same syntax as the
relational algebra and has following semantics:
\begin{align*}
  \denote{\emptyset}(I)(t) &= 0 \\
  \denote{\set{t}}(I)(u) &= \begin{cases}
    1 & t = u \\
    0 & t \neq u
  \end{cases} \\
  \denote{R}(I)(t) &= I(R)(t) \\
  \denote{\select_\phi(Q)}(I)(t) &= \phi(t) \cdot \denote{Q}(I)(t) \\
  \denote{\project_A(Q)}(I)(t) &= \sum_{u, u[A] = t} \denote{Q}(I)(u)\\
  \denote{Q_1 \join Q_2}(I)(t) &= \denote{Q_1}(I)(t[A_1]) \cdot \denote{Q_2}(I)(t[A_2]) \\
  \denote{Q_1 \cup Q_2}(I)(t) &= \denote{Q_1}(I)(t) + \denote{Q_2}(I)(t) \\
  \denote{\rename_{A \mapsto B}(Q)}(I)(t) &= \denote{Q}(I)(t[B \mapsto A])
\end{align*}

Note that
\begin{itemize}
  \item
    the semiring $(\set{\bot, \top}, \lor, \land, \bot, \top)$ corresponds to
    relational algebra,
  \item
    the semiring $(\nats, +, \times, 0, 1)$ corresponds to bag algebra,
  \item
    the semiring $(2^{TupleId}_\bot, \cup_L, \cup_S, \bot, \emptyset)$
    corresponds to lineage,
  \item
    the semiring $(2^{2^{TupleId}}, \cup, \Cup, \emptyset, \set{\emptyset})$
    corresponds to why-provenance, and
  \item
    the semiring $(\nats[TupleId], +, \times, 0, 1)$ corresponds to
    how-provenance.
\end{itemize}

Given an assignment $v: TupleId \to K$ from tuple ids to elements of another
semiring $K$, evaluating the how-provenance polynomial for a query $Q$ (after
substituting $v(x)$ for $x$) simulates evaluating $Q$ under $K$. For example,
given a how-provenance polynomial, we can use $v(x) = \top$ for relational
algebra, $v(x) = 1$ for bag algebra, $v(x) = \set{x}$ for lineage, and $v(x) =
\set{\set{x}}$ for why-provenance.

\section{Datalog Provenance}
- Provenance graphs for monotone datalog.
- Translating provenance graphs to how-provenance, why-provenance, and lineage.
- Extending provenance and provenance graphs to datalog with negation and
  aggregation.

page 89
``
Both lineage and Trio lineage have been defined for non-monotone operations
such as aggregation and negation operations. However, it is non-obvious how to
extend these definitions to why- or how-provenance while preserving their nice
properties
''

\section{Dedalus Provenance}
- Provenance for Dedalus programs. EDB represents network requests.

\section{Network Provenance}
- Network provenance ala Boon.

\section{Wat-Provenance}
- Our new form of provenance.

\end{document}
